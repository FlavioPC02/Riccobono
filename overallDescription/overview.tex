Here you can see how to include an image in your document.

\begin{sidewaysfigure}
\centering
\includegraphics[width=\textwidth]{Images/11.png}
\caption{\label{fig:metamodel}DICE DPIM metamodel.}
\end{sidewaysfigure}

\begin{figure}
\centering
\includegraphics[width=\textwidth]{Images/11.png}
\caption{\label{fig:metamodel2}DICE DPIM metamodel in portrait form.}
\end{figure}

\subsection{Product perspective}
\subsubsection{Scenarios}

\paragraph{Scenario 1: Access as a non-registered user}
A cyclist who has never used BBP before opens the application without creating an account. From the homepage, the user can explore the map of the metropolitan area and inspect the available bike paths, viewing information such as their approximate location and current status. The user can also access general functionalities offered by BBP, such as searching for a route between an origin and a destination, although 
with limitations: they cannot record trips, save personal data or contribute new information unless they later decide to register.

\paragraph{Scenario 2: User registration}
A cyclist decides to create an account on BBP in order to record their trips and contribute information about bike paths. After accessing BBP, the user chooses the option to register and is asked to provide the required personal data and credentials. The user enters the requested information, such as an e-mail address and a password, and confirms the registration. BBP validates the provided data and creates a new registered user associated with that information. From this moment on, the cyclist can log in using their credentials and access additional functionalities, such as trip recording and the insertion of information about bike paths.

\paragraph{Scenario 3: Record a trip and view statistics}
A registered user wants to keep track of a new cycling activity. After logging in to BBP, the user decides to record a trip before starting to ride. The system starts tracking the trip. When the ride is over, the user stops the recording. BBP stores the trip as part of the user’s personal history and computes statistics such as total distance and average speed. The user can immediately view the summary of the recorded trip and can access their list of past trips to review these statistics and monitor their cycling activities over time.

\paragraph{Scenario 4: Insert and (optionally) publish information about a bike path}
After finishing the trip, the registered user logs in to BBP and searches for the corresponding bike path in the system. BBP shows the basic information currently associated with that path. The user then chooses to add new information, including their status (e.g., optimal, medium, sufficient, requires maintenance, …) and the presence of relevant
obstacles, for instance, potholes. Before storing the contribution, the BBP allows the user to decide whether this information should be made publishable to other users. If the user confirms publication, the new information becomes visible to all users when they 
inspect that bike path. If the user decides not to publish it, the information is stored, but is not shown as part of the public data available to the community.

\paragraph{Scenario 5: Search for a route between an origin and a destination}
A user, either registered or not, wants to find a suitable route to travel by bike between two locations. The user specifies an origin and a destination, and BBP analyses the available bike paths and other relevant information to compute one or more bike-friendly routes connecting the two points. The system then presents these routes on a map, together with details such as their approximate length and the 
status of the bike paths involved. The user compares the suggested routes and selects the one they consider most appropriate for their needs.

\subsubsection{Class Diagram}
\subsubsection{State Diagram}

\subsection{Product functions}
\subsection{User characteristics}
\subsection{Assumptions, dependencies and constraints}